\documentclass[11pt,a4paper]{article}
\usepackage[utf8]{inputenc}
\usepackage[T1]{fontenc}
\usepackage{lmodern}
\usepackage[french]{babel}
\usepackage[hidelinks]{hyperref}
%\usepackage{hyperref}
\usepackage{url}
\usepackage{cite}
\usepackage[babel=true]{csquotes} 
\usepackage{amsmath}
\DeclareMathOperator{\arcsinh}{arcsinh}
\usepackage{amsthm}
\usepackage{amssymb}
\usepackage[version=3]{mhchem}
\usepackage{vmargin}
\usepackage{pgfplots}
\usepackage{enumerate}

%\usepackage{subfig}
% subfig is deprecated see
% http://en.wikibooks.org/wiki/LaTeX/Floats,_Figures_and_Captions#Subfloats
\usepackage{caption}
\usepackage{subcaption}

% Numbers and units
\usepackage[squaren, Gray]{SIunits}
\usepackage{sistyle}
\usepackage[autolanguage]{numprint}
%\usepackage{numprint}
\newcommand\si[2]{\numprint[#2]{#1}}
\newcommand\np[1]{\numprint{#1}}

\usepackage{ifthen}
\usepackage{tikz}
\usepackage{verbatim}
\usetikzlibrary{calc,arrows}
\usepackage[framemethod=tikz]{mdframed}
%http://tex.stackexchange.com/questions/50877/excursus-environment-using-mdframed-issue-with-page-breaks
\tikzset{
   excursus arrow/.style={%
    line width=2pt,
    draw=gray!40,
    rounded corners=2ex,
   },
   excursus head/.style={
    fill=white,
    font=\bfseries\sffamily,
    text=gray!80,
    anchor=base west,
   },
}

\mdfdefinestyle{mysquare}{%
  singleextra={%
   \path let \p1=(P), \p2=(O) in (\x2,\y1) coordinate (Q);%
   \path let \p1=(Q), \p2=(O) in (\x1,{(\y1-\y2)/2}) coordinate (M);%
   \path [excursus arrow, round cap-to]%
   ($(O)+(5em,0ex)$) -| (M) |- %
   ($(Q)+(12em,0ex)$) .. controls +(0:16em) and +(185:6em) .. %
   ++(23em,2ex);%
   \node [excursus head] at ($(Q)+(2.5em,-0.75pt)$) {Solution};},
  firstextra={%
   \path let \p1=(P), \p2=(O) in (\x2,\y1) coordinate (Q);
   \path [excursus arrow,-to]
   (O) |- %
   ($(Q)+(12em,0ex)$) .. controls +(0:16em) and +(185:6em) .. %
    ++(23em,2ex);
   \node [excursus head] at ($(Q)+(2.5em,-2pt)$) {Solution};
  },
  secondextra={%
   \path let \p1=(P), \p2=(O) in (\x2,\y1) coordinate (Q);
   \path [excursus arrow,round cap-]
   ($(O)+(5em,0ex)$) -| (Q);
  },
  middleextra={%
   \path let \p1=(P), \p2=(O) in (\x2,\y1) coordinate (Q);
   \path [excursus arrow](O) -- (Q);
 },
 middlelinewidth=2.5em,middlelinecolor=white,
 hidealllines=true,topline=true,
 innertopmargin=0.5ex,
 innerbottommargin=2.5ex,
 innerrightmargin=2pt,
 innerleftmargin=2ex,
 skipabove=0.87\baselineskip,
 skipbelow=0.62\baselineskip,
}


\ifthenelse{\isundefined{\Sol}}{\def\Sol{true}}{}
\ifthenelse{\equal{\Sol}{false}}
{
  \newenvironment{solution}{\expandafter\comment}{\expandafter\endcomment}
}
{
  \newmdenv[style=mysquare]{solution}
}


%\newcommand{\solution}[1]
%{\ifthenelse{\equal{\Sol}{true}}{\subsection*{Solution}#1}{}}

\DeclareMathOperator{\newdiff}{d} % use \dif instead
\newcommand{\dif}{\newdiff\!}
\newcommand{\fpart}[2]{\frac{\partial #1}{\partial #2}}
\DeclareMathOperator{\res}{Res}
\DeclareMathOperator{\arctanh}{arctanh}
\newcommand{\ffpart}[2]{\frac{\partial^2 #1}{\partial #2^2}}
\newcommand{\fdpart}[3]{\frac{\partial^2 #1}{\partial #2\partial #3}}
\newcommand{\fdif}[2]{\frac{\dif #1}{\dif #2}}
\newcommand{\ffdif}[2]{\frac{\dif^2 #1}{\dif #2^2}}
\newcommand{\constant}{\ensuremath{\mathrm{cst}}}

\newcommand{\R}{\ensuremath{\mathbb{R}}}
\newcommand{\Rn}{\R^n}

\newcommand{\HRule}{\rule{\linewidth}{0.5mm}}

\newcommand{\hypertitle}[4]{
\usepackage{hyperref}
{\renewcommand{\and}{\unskip, }
\hypersetup{pdfauthor={#4},
            pdftitle={INMA1170 : Correction de l'examen des #3 de #2 #1},
            pdfsubject={Analyse num\'erique}}
}
\ifthenelse{\equal{\Sol}{true}}%
{\title{INMA 1170 : Correction de l'examen des #3 de #2 #1}}%
{\title{INMA 1170 : Énoncé de l'examen des #3 de #2 #1}}
\author{#4}

\begin{document}

\maketitle

\paragraph{Informations importantes}
Ce document a été réalisé par les étudiants susnommés et
est donc à prendre avec des ``pincettes''.
Si vous observez des fautes ou avez des suggestions à faire,
n'hésitez pas à nous contacter.
On peut retrouver le code source à l'adresse suivante
\begin{center}
  \url{https://github.com/blegat/linma1170-examens}.
\end{center}
On y trouve aussi le contenu du \texttt{README} qui contient de plus
amples informations, vous êtes invité à le lire.
Il y a aussi ce post
\begin{center}
  \url{http://www.forum-epl.be/viewtopic.php?p=109318}.
\end{center}
qui parle de ce recueil d'examens.
}


\hypertitle{2013}{Janvier}{majeures}{Legat Beno\^it}

\paragraph{Ressource utile}
\url{http://www.forum-epl.be/viewtopic.php?t=12324}

\section*{Question 1 (4 pts)}
Démontrez le théorème de Laguerre:
\textit{
  Soit le polynôme à coefficients complexes
  $p(z) = z^n + a_{n-1}z^{n-1} + \cdots + a_0$.
  Soit $z_0$ tel que $p(z_0)p'(z_0) \neq 0$.
  Alors, il existe au moins une racine du polynôme dans le disque
  fermé $z_0$ et de rayon $R = n\frac{|p(z_0)}{p'(z_0)}$.
}

Donnez la démonstration seulement dans le cas de zéros distincts

\solution{
}

\section*{Question 2 (4 pts)}
Démontrez le théorème suivant (variant des notes de cours):
\textit{
  Si $A$ est une matrice strictement diagonalement dominante,
  la méthode de Jacobi est convergente.
}
(Rappel: pour la méthode de Jacobi, $M = D$ et $N = -(L+U)$.)

\solution{
}

\section*{Question 3 (4 pts)}
Démontrer le théorème suivant:
\textit{
  Si $f: \Rn \to \Rn$ a un point fixe $s \in \Rn$ et si $f$ satisfait
  la condition de Lipschitz
  \begin{align*}
    \|f(x) - f(s)\| & \leq L\|x-s\|, &
    \forall x \in B_\epsilon(s) & := \{x \in \Rn : \|x - s\| < \epsilon\}
  \end{align*}
  alors tous les itérés de $x_{k+1} = f(x_k)$ appartiennent à $B_\epsilon(s)$
  si $x_0 \in B_\epsilon(s)$ et convergent exponentiellement vers $s$,
  qui est le seul point fixe de $f$ dans $B_\epsilon(s)$.
}

\solution{
}

\section*{Question 4 (4 pts)}
Donnez la méthode de l'itération inverse pour une matrice $A$
diagonalisable et avec valeurs propres $\lambda_j$,
quand on utilise un shift $\mu$ pour calculer le vecteur propre
correspondant à la valeur propre $\lambda_J$ la plus proche de $\mu$.
Discutez de sa convergence quand $|\lambda_j - \mu|$ est
strictement plus petit que tout autre $|\lambda_j - \mu|$, $j \neq J$.

\solution{
}

\section*{Question 5 (4 pts)}
Définissez le concept de région de stabilité d'une méthode numérique
d'une équation différentielle ordinaire et expliquez le concept de
stabilité absolue.

\solution{
}

Essayez de limiter vos réponses à 2 pages par question (soyez concis).

\end{document}
